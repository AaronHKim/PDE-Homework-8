\documentclass[12pt,letterpaper]{article}
\usepackage{preamble}
\usepackage{siunitx}
\usepackage{adjustbox}
\usepackage{enumitem}
%%%%%%%%%%%%%%%%%%%%%%%%%%%%%%%%%%%%%%%%%%
%%%% Edit These for yourself
%%%%%%%%%%%%%%%%%%%%%%%%%%%%%%%%%%%%%%%%%%

\newcommand{\mathsym}[1]{{}}
\newcommand{\unicode}[1]{{}}

\newcommand\course{Graduate Modelling 2 MAT 5460}
\newcommand\hwnumber{5}
\newcommand\userID{Aaron Kim}
\def\changemargin#1#2{\list{}{\rightmargin#2\leftmargin#1}\item[]}
\let\endchangemargin=\endlist 

\theoremstyle{definition}
\newtheorem*{thm}{Theorem}

\renewcommand\thesection{\arabic{section}}
\renewcommand\thesubsection{\thesection.\arabic{subsection}}



\begin{document}
\section*{Problem 26}

Show that the Casoratian $W(n)$ given by:
\begin{equation}\label{eq:cassoratian}
    W(n) = \left|\begin{matrix}
        x_1(n) & x_2(n) &  \cdots & x_k(n) \\
        x_1(n+1) & x_2(n+1) & \cdots & x_k(n+1)\\
        \vdots & & & \vdots\\
        x_1(n+k-1) & x_2 (n+k-1) & \cdots & x_k(n+k-1)
    \end{matrix}\right|
\end{equation}
may be given by the formula:
\begin{equation*}
    W(n) = \left|\begin{matrix}
        x_1(n) & x_2(n) & \cdots & x_k(n)\\
        \Delta x_1(n) & \Delta x_2(n) & \cdots & \Delta x_k(n)\\
        \vdots & & & \vdots \\
        \Delta^{k-1}x_1(n) & \Delta^{k-1} x_2 (n) & \cdots & \Delta^{k-1} x_k(n)
    \end{matrix}\right|
\end{equation*}



\textbf{Attempt:}
\begin{changemargin}{.5cm}{0cm}

We know by definition that,  $\Delta x_i(n)=x_i(n+1)- x_i(n)$ , and that $\Delta x_i(n+k-1)=x_i(n+k)-x_i(n+k-1)$. We also know that $E= \Delta + I$, where $x_i(n+k-1) = E^{k-1}x_i(n) =\sum_{i=0}^{k-1} (-1)^{k-1} {k-1 \choose i} \Delta^{k-1-i}x(n)$. Then we can rewrite (\ref{eq:cassoratian}) as:

\begin{equation*}
    W(n) = \left|\begin{matrix}
        x_1(n) & x_2(n) &  \cdots & x_k(n) \\
        Ex_1(n) & Ex_2(n) & \cdots & Ex_k(n)\\
        \vdots & & & \vdots\\
        E^{k-1}x_1(n) & E^{k-1}x_2 (n) & \cdots & E^{k-1}x_k(n)
    \end{matrix}\right|
\end{equation*}

We can observe that for the $(2\times2)$ case:
\begin{equation*}
    W(n) = \left|\begin{matrix}
        x_1(n) & x_2(n)  \\
        Ex_1(n) & Ex_2(n) 
    \end{matrix}\right|= \left|\begin{matrix}
        x_1(n) & x_2(n)  \\
        \Delta x_1(n)+x_1(n) & \Delta x_2(n) +x_2(n)
    \end{matrix}\right|
\end{equation*}
Where it is clear that the repeated term in each row $x_1(n)$ and $x_2(n)$ will cancel i.e. using the row operation $R_2-R_1 \rightarrow R_2$ we get: 
\begin{equation*}
    W(n) = \left|\begin{matrix}
        x_1(n) & x_2(n)  \\
        Ex_1(n) & Ex_2(n) 
    \end{matrix}\right|= \left|\begin{matrix}
        x_1(n) & x_2(n)  \\
        \Delta x_1(n) & \Delta x_2(n) 
    \end{matrix}\right|
\end{equation*}
Similarly for the $(3\times3)$ we get: \begin{align*}
    W(n) &= \left|\begin{matrix}
        x_1(n) & x_2(n) & x_3(n) \\
        Ex_1(n) & Ex_2(n) & Ex_3(n)\\
        E^2x_1(n) & E^2x_2(n) & E^2x_3(n)
    \end{matrix}\right|\\
    &= \left|\begin{matrix}
         x_1(n) & x_2(n) & x_3(n) \\
        \Delta x_1(n) + x_1(n) & \Delta x_2(n)+ x_2(n) & \Delta x_3(n)+ x_3(n)\\
        \Delta^2 x_1(n)+ 2\Delta x_1(n) + x_1(n) & \cdots & \Delta^2 x_3(n)+ 2\Delta x_3(n) + x_3(n)
    \end{matrix}\right|
\end{align*}
So if we then perform the row operations again as before, $R_2-R_1 \rightarrow R_2$, $R_3-R_1 \rightarrow R_3$:
\begin{align*}
    \left|\begin{matrix}
         x_1(n) & x_2(n) & x_3(n) \\
        \Delta x_1(n)  & \Delta x_2(n)& \Delta x_3(n)\\
        \Delta^2 x_1(n)+ 2\Delta x_1(n)  & \cdots & \Delta^2 x_3(n)+ 2\Delta x_3(n) 
    \end{matrix}\right|
\end{align*}
And we can then perform a similar row operation from $R_3-2R_2 \rightarrow R_3$
\begin{align*}
    \left|\begin{matrix}
         x_1(n) & x_2(n) & x_3(n) \\
        \Delta x_1(n)  & \Delta x_2(n)& \Delta x_3(n)\\
        \Delta^2 x_1(n)  & \cdots & \Delta^2 x_3(n) 
    \end{matrix}\right|
\end{align*}
\end{changemargin}

And in general, we can perform these row ops up to $k-1$ from the summation $sum_{i=0}^{k-1} (-1)^{k-1} {k-1 \choose i} \Delta^{k-1-i}x(n)$, since each row term will cancel a multiple of the one before it, except the first term, $\Delta^{k-1} x_i(n)$, so that we get the final form:

\begin{equation*}
    W(n) = \left|\begin{matrix}
        x_1(n) & x_2(n) & \cdots & x_k(n)\\
        \Delta x_1(n) & \Delta x_2(n) & \cdots & \Delta x_k(n)\\
        \vdots & & & \vdots \\
        \Delta^{k-1}x_1(n) & \Delta^{k-1} x_2 (n) & \cdots & \Delta^{k-1} x_k(n)
    \end{matrix}\right|
\end{equation*}

\newpage

\section*{Problem 27}

Consider the second order difference equation 
\begin{equation*}
    u(n+2) - \dfrac{n+3}{n+2}u(n+1) + \dfrac{2}{n+2}u(n) = 0
\end{equation*}

\begin{enumerate}[label = (\alph*)]
    \item show that $\displaystyle u_1(n)= \dfrac{2^n}{n!}$ is a solution of the equation.
    \item Use the above formula to find another solution $u_2(n)$ of the equation.
\end{enumerate}

\textbf{Attempt:}
\begin{changemargin}{.5cm}{0cm}

\begin{enumerate}[label=(\alph*)]
    \item To check that this is a solution: \begin{align*}
        &=\dfrac{2^{n+2}}{(n+2)!} - \dfrac{n+3}{n+2}\dfrac{2^{n+1}}{(n+1)!} + \dfrac{2}{n+2}\dfrac{2^n}{n!}\\
        &=\dfrac{2^{n+2}}{(n+2)!} - \dfrac{(n+3)2^{n+1}}{(n+2)!} + \dfrac{2^{n+1}(n+1)}{(n+2)!}\\
        &=\dfrac{2^{n+1}}{(n+2)!} \left(2- (n+3) + n+1\right)\\
        &=\dfrac{2^{n+1}}{(n+2)!}(0)\\
        &=0
    \end{align*}
    So we know that this is a solution for our equation.
    \item We know from the notes that, if $u_1$ and $u_2$ are solutions of the equation and $W(n)$ is their Casoratian, then: 
    \begin{equation*}
        u_2(n) = u_1(n) \left[\sum_{r=0}^{n-1}\frac{W(r)}{u_1(r)u_1(r+1)} \right]
    \end{equation*}
    is a solution to this equation. Using Abel's lemma,
    
    \fbox{\begin{minipage}{\textwidth}\textbf{Abel's Lemma:} 

Let $x_1(n),x_2(n),\ldots,x_k(n)$ be solutions of:
\begin{equation*}
    x(n+k)+p_1(n)x(n+k-1)+\cdots+p_k(n)x(n)=0
\end{equation*} and let $W(n)$ be their Casoratian. Then, for $n\geq n_0$,
\begin{equation*}
    W(n) = (-1)^{k(n-n_0)}\left(\prod_{i=n_0}^{n-1}p_k(i) \right)W(n_0)
\end{equation*}

\end{minipage}}
    
    
    If we choose a constant $W(r_0)=1$:
    \begin{equation*}
        W(r) = (-1)^{2(r-r_0)}\left(\prod_{i=r_0}^{r-1}p_2(i) \right)
    \end{equation*}
    Then we know the form of our polynomial has two coefficients, $k=1,k=2$. If we choose $k=2$ we know that an even number times any other number will be even, the term $(-1)^{2(r-r_0)}=1$, giving us:
    \begin{equation*}
        W(r) =\left(\prod_{i=r_0}^{r-1}\dfrac{2}{i+2} \right)
    \end{equation*}
    So this gives us:
    \begin{equation*}
        u_2(n) = \dfrac{2^n}{n!} \left[\sum_{r=0}^{n-1}\frac{\left(\prod_{i=0}^{r-1}\dfrac{2}{i+2} \right)}{\left(\dfrac{2^r}{r!}\right)\left(\dfrac{2^{r+1}}{(r+1)!}\right)} \right]
    \end{equation*}
    \begin{equation*}
        u_2(n) = \dfrac{2^n}{n!} \left[\sum_{r=0}^{n-1}\frac{\left(\dfrac{2^{r}}{(r+1)!} \right)}{\left(\dfrac{2^r}{r!}\right)\left(\dfrac{2^{r+1}}{(r+1)!}\right)} \right]
    \end{equation*}
    \begin{equation*}
        u_2(n) = \dfrac{2^n}{n!} \left[\sum_{r=0}^{n-1}\dfrac{r!}{2^{r+1}} \right]
    \end{equation*}
\end{enumerate}

\end{changemargin}
\newpage


\section*{Problem 28}

Write the general solution of the difference equation:
\begin{equation*}
    (E-3)^2(E^2+4)x(n)=0
\end{equation*}



\textbf{Attempt:}

    \begin{changemargin}{.5cm}{0cm}
        We can rewrite the above as:
        \begin{equation*}
            (\lambda-3)^2(\lambda^2+4)=0
        \end{equation*}
        The characteristic rots give us the general solution:
        \begin{align*}
            x(n) = a_03^n + a_1n3^n + b_1(2i)^n+b_2(-2i)^n
        \end{align*}

        
    \end{changemargin}





\newpage

%

\section*{Problem 29}

Find a homogenous difference equation whose solution is:
\begin{equation*}
    y(n) = 2^{n-1}-5^{n+1}
\end{equation*}


\textbf{Attempt:}

\begin{changemargin}{.5cm}{0cm}
    We know that the solution to our equation is given above as:
    \begin{equation}\label{eq:p29}
       y(n) = 2^{n-1}-5^{n+1}
    \end{equation}
    Then we know also that the solution should have the form:
    \begin{equation*}
        y(n) = a_1(\lambda_1)^n+a_2(\lambda_2)^n 
    \end{equation*}
    So we know that we can rewrite (\ref{eq:p29}) as:
    \begin{equation*}
        y(n)=\left(\dfrac{1}{2}\right)^2(2)^n + (-5)(5)^n
    \end{equation*}
    Then $\lambda_1 = 2$, and $\lambda_2 = 5$, and we can get the difference equation:
    \begin{equation*}
        (E-2)(E-5)x(n)=0
    \end{equation*}
\end{changemargin}

\newpage

\section*{Problem 30}

Find a particular solution of the difference equation:
\begin{equation*}
    x(n+2)-5x(n+1)+4x(n)=4^n-n^2
\end{equation*}


    
  \textbf{Attempt:}
    \begin{changemargin}{.5cm}{0cm}
    
    We can rewrite the above as:
    \begin{equation*}
        (E^2-5E+4)(x(n)=4^n-n^2
    \end{equation*}
    \begin{equation*}
        (E-4)(E-1)x(n)=4^n-n^2
    \end{equation*}
    
    So this gives us:
    \begin{equation*}
        \lambda_1=4,\lambda_2=1
    \end{equation*}
    Which gives us :
    \begin{equation*}
    y_c(n)=c_1(4)^n + c_2(1)^n    
    \end{equation*}
    we know that $g(n) = 4^n - n^2$, and we know that $(E-4)$ annihilates $4^n$, so we have at least one $\lambda_1 = \mu_1$ so we are in case 2. To annihilate $n^2$
    \end{changemargin}


\end{document}